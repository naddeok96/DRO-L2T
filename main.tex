\documentclass[11pt]{article}

% ---------- Preamble ----------
\usepackage[margin=1in]{geometry}
\usepackage{amsmath,amssymb,amsthm,mathtools}
\usepackage{booktabs}
\usepackage{graphicx}
\usepackage{float}
\usepackage{natbib}
\usepackage{hyperref}
\usepackage{url}
\usepackage{enumitem}
\usepackage{algorithm}
\usepackage{algpseudocode}
\usepackage{xcolor}
\usepackage{subcaption}

\hypersetup{
  colorlinks=true,
  linkcolor=black,
  citecolor=blue,
  urlcolor=blue
}

\newtheorem{definition}{Definition}
\newtheorem{proposition}{Proposition}
\newtheorem{assumption}{Assumption}

\newcommand{\MacroAcc}{\mathrm{MacroAcc}}
\newcommand{\E}{\mathbb{E}}
\newcommand{\R}{\mathbb{R}}
\newcommand{\1}{\mathbf{1}}
\newcommand{\Prb}{\mathbb{P}}

\title{\textbf{Learning How to Teach: \\Reinforcement Learning for Universal Curriculum Design Across Tasks and Datasets}}

\author{Anonymous Authors}
\date{}

% ---------- Document ----------
\begin{document}
\maketitle

\begin{abstract}
Modern ML systems increasingly learn from multiple heterogeneous datasets and across task families (e.g., classification, detection, sequence labeling, QA), where naive sampling can overfit to large or easy sources and waste budget. We revisit \emph{learning to teach}—training a reinforcement-learning (RL) teacher to control a student's learning dynamics—and argue that three properties are essential but have not been addressed together: (i) a \emph{Markov} teacher–student formulation whose next-step outcomes depend only on the current state and action, (ii) \emph{number-of-datasets invariance} so one policy handles any $N$ and any dataset permutation, and (iii) \emph{task- and model-invariance} so the same policy transfers across task families and architectures. Prior learning-to-teach work has demonstrated per-example data selection and dynamic loss design \citep{fan2018l2t,wu2018l2tloss}, but typically without a formal Markov interface, without an explicitly \emph{set-invariant} treatment of many datasets, and without a small, task/architecture/optimizer-agnostic observation.
We present a Markov teacher--student MDP grounded in SGD dynamics and introduce a \emph{grouped, universal state}: (A) \textbf{dataset-level DeepSets embeddings} built from permutation-invariant statistics of predictions, losses, gradients, and diversity; (B) a \textbf{model-complexity block} with size/compute and capacity proxies; and (C) a \textbf{training-progress block} with normalized, task-agnostic dynamics signals. This state supports a set-based encoder and shared per-dataset policy head with mixture outputs valid for any $N$. The teacher jointly controls dataset mixture, learning rate, and usage within a CMDP budget view. We validate (a) \emph{approximate Markovity} via one-step prediction $R^2$ and small gains from adding short history, (b) invariance under varying $N$ and permutations, and (c) \emph{cross-task/architecture transfer}, including leave-one-task-family-out and zero-/few-shot transfer from toy episodes to large models. Despite its compactness, the grouped state yields strong sample efficiency and balanced performance (macro-accuracy / worst-dataset) across heterogeneous datasets and tasks, unifying Markovity, set invariance, and portability.
\end{abstract}

\section{Introduction}
Curriculum learning posits that \emph{what} a learner studies \emph{when} matters for both efficiency and generalization \citep{bengio2009curriculum,kumar2010selfpaced}. In today’s practice, models often draw from multiple domains with disparate difficulty, size, and noise; naive sampling overemphasizes large or easy sources and can degrade worst-case performance. We study \emph{learning to teach}: a reinforcement-learning (RL) \emph{teacher} that observes the student’s competence and adaptively chooses where to train next---not only which data to present, but also how aggressively to spend budget and at what learning rate---to maximize balanced, end-of-training objectives such as \emph{macro-accuracy} across $N$ datasets.

A central obstacle is the design of the teacher’s state and interface. For well-posed RL, we would like a \emph{Markov} formulation in which next-step outcomes depend only on the current state and action (not long histories). For scalability, we require \emph{number-of-datasets invariance}: a single policy that handles any $N$ and any permutation of datasets without architectural changes or scale drift. And for portability, we seek \emph{task/architecture invariance}: an observation that is agnostic to task semantics so the same teacher can be meta-trained on inexpensive episodes and \emph{transferred} to large models across modalities (classification, detection, NER, QA, etc.).

Prior learning-to-teach works demonstrate the promise of RL teachers for \emph{data selection} \citep{fan2018l2t} and dynamic loss design \citep{wu2018l2tloss}, and multi-task scheduling and loss-weighting methods can improve joint training \citep{kendall2018uncertainty,chen2018gradnorm}. However, existing approaches typically (i) omit a formal Markov teacher--student interface and do not diagnose when compact observations suffice; (ii) do not treat the state as a \emph{set} with explicit permutation and cardinality invariance; and (iii) rely on task-specific signals or hand-crafted features that do not readily transfer across task families. Meanwhile, meta-RL frameworks \citep{duan2016rl2,finn2017maml} and permutation-invariant encoders for sets \citep{zaheer2017deepsets,lee2019settransformer} provide the ingredients to address these desiderata, but have not been combined in a unified, practical teacher for multi-dataset, multi-task training.

\textbf{Our approach.} We formalize the hidden state of training under SGD and define a task/model/optimizer-agnostic observation split into three groups: \emph{(A) dataset-level DeepSets embeddings}, \emph{(B) model complexity}, and \emph{(C) training progress}. (A) is computed only from model probabilities and update magnitudes on tiny probes and includes loss/uncertainty histograms, confidence percentiles, gradient-norm moments, diversity sketches, and light dataset meta-statistics (cf. the student-feedback features in \citet{fan2018l2t}; and the loss-shaping view of \citet{wu2018l2tloss}). (B) summarizes capacity with log-parameters, FLOPs, overfit gap, and optimizer hyperparameters. (C) tracks normalized time/budget, EMAs of train/val losses and their gaps, gradient dynamics, and LR ratio. This universal grouped state feeds a \emph{set-based} encoder (DeepSets with mean pooling) with shared per-dataset policy heads whose masked-softmax mixture is valid for any $N$ and any dataset ordering. The teacher jointly controls dataset mixture, learning rate, and budget usage within a CMDP and is trained with PPO and a dual variable. We \emph{validate} approximate Markovity via one-step prediction and minimal history gains, show robustness to changes in $N$ and permutations, and demonstrate cross-task/architecture transfer, aligning with and extending \citet{fan2018l2t,wu2018l2tloss}.

\textbf{Contributions.} (1) A Markov teacher--student formulation grounded in SGD with a \emph{grouped, task/architecture-agnostic} state: dataset-level DeepSets embeddings + model-complexity block + training-progress block, with empirical diagnostics of approximate Markovity. (2) A \emph{number-of-datasets invariant} policy/value parameterization via set encoders and shared per-dataset policy heads, yielding valid mixture actions for any $N$ and any permutation. (3) A cross-task and cross-architecture meta-training/evaluation protocol (including leave-one-task-family-out and zero-/few-shot transfer) demonstrating \emph{portability}. (4) A CMDP budget view with mixture, learning-rate, and usage control, improving \emph{balanced} performance (macro-accuracy and worst-dataset metrics) over uniform, myopic/bandit, and schedule controllers.

\textbf{Why this matters.} Practitioners increasingly train models on \emph{many} datasets under tight budgets and deployment constraints. A teacher that is \emph{Markov}, \emph{set-invariant}, and \emph{task/architecture-invariant} can be trained once on cheap episodes and ported across modalities and scales, providing balanced performance and sample efficiency without bespoke engineering for each new mixture. Our results suggest that a compact, grouped observation (dataset embeddings + model/training blocks) is sufficient to steer learning dynamics effectively, enabling portable curriculum policies that complement existing data-selection, loss-weighting, and hyperparameter schedulers.

\section{Related Work}
\paragraph{Learning to teach and curriculum RL.} The idea of using a teacher to shape a learner’s experience dates to classic curriculum and self-paced learning \citep{bengio2009curriculum,kumar2010selfpaced}. \citet{fan2018l2t} cast \emph{Learning to Teach} as an RL problem in which a teacher selects which training instances to present; they report generalization of the teacher across architectures and datasets. \citet{wu2018l2tloss} learn dynamic loss functions to guide training. These works focus on \emph{per-example selection or loss design}, and, to our knowledge, do not formalize a Markov teacher--student interface, do not provide \emph{set-invariant} processing across a variable number of datasets, and do not propose a compact, \emph{task-agnostic} observation that transfers across modalities.

\paragraph{Multi-task scheduling and loss weighting.} Scheduling across tasks and weighting losses can substantially affect multi-task outcomes. Uncertainty-based weighting \citep{kendall2018uncertainty} and gradient-balancing methods such as GradNorm \citep{chen2018gradnorm} adapt losses to stabilize training. These approaches are effective but do not explicitly reason about Markovian state, budgeted control, or variable-cardinality sets of datasets; our teacher can complement them by controlling dataset mixture and budget usage.

\paragraph{Meta-learning and RL controllers.} Meta-RL and learning-to-learn methods \citep{duan2016rl2,finn2017maml} optimize learning procedures but typically assume either full recurrent memory or task-specific state. Hyperparameter controllers such as PBT and BOHB \citep{jaderberg2017pbt,falkner2018bohb,snoek2012practical} automate schedules, yet they lack a formal, task-agnostic state and do not guarantee invariance in $N$.

\paragraph{Permutation-invariant encoders for sets.} DeepSets \citep{zaheer2017deepsets} establish a universal form for permutation-invariant functions over sets, and Set Transformer extends this with attention-based pooling \citep{lee2019settransformer}. We adopt these encoders for both state and action spaces, yielding policies whose per-dataset mixture logits remain valid for \emph{any} $N$ and any permutation, a property absent from most teaching/scheduling work.

\paragraph{Data selection, uncertainty, and prioritized replay.} Uncertainty-based selection, influence/marginal-utility estimators, and prioritized replay in RL \citep{schaul2016prioritized} adapt which samples are emphasized during training. Our work differs in objective and scope: we aim for \emph{balanced final performance} across multiple datasets under a budget, with the teacher controlling mixtures, step size, and usage from a compact, task-agnostic state.

\medskip
\noindent\textbf{Position relative to prior work.} Compared to \citet{fan2018l2t} and \citet{wu2018l2tloss}, we \emph{(i)} formalize a Markov teacher--student MDP and validate \emph{approximate Markovity} of a compact, universal \emph{grouped} observation; \emph{(ii)} enforce \emph{number-of-datasets invariance} by representing the state as a set and sharing per-dataset policy heads; and \emph{(iii)} target \emph{task/architecture invariance} by using only task-agnostic statistics (probabilities, losses, gradient magnitudes, diversity sketches) and lightweight capacity/progress summaries. This combination---Markovity, set invariance, and portability---is central to practicality.


\section{Problem Formulation}
We consider $N$ datasets $\{\mathcal{D}_i\}_{i=1}^N$ drawn from an episode-specific task $\tau\sim\mathcal{T}$.
A student model $f_\theta$ with parameters $\theta$ is trained under a teacher policy $\pi_\phi$ over horizon $H$ with initial budget $B_0$.

\subsection{Markov Teacher--Student MDP}\label{sec:mdp}
\paragraph{Definition (Markov property).}
An RL environment with observable state $S_t$ and action $a_t$ is Markov if
\begin{equation}
\Prb(S_{t+1},r_t \mid S_t,a_t) \;=\; \Prb(S_{t+1},r_t \mid S_t,a_t,S_{t-1},a_{t-1},\dots).
\label{eq:markov}
\end{equation}

\begin{definition}[Approximate Markovity]
\label{def:approx-markov-tight}
Let $\mathcal{F}$ be a rich predictor class. Define
$\mathcal{E}_0 \triangleq \inf_{F\in\mathcal{F}} \E\| S_{t+1} - F(S_t,a_t)\|^2$
and
$\mathcal{E}_1 \triangleq \inf_{F\in\mathcal{F}} \E\| S_{t+1} - F(S_t,a_t,S_{t-1},a_{t-1})\|^2$.
We say $g$ is $(\varepsilon,\delta)$-Markov if $\mathcal{E}_0 \le \varepsilon$ and
$\mathcal{E}_0 - \mathcal{E}_1 \le \delta$ with $\delta \ll \varepsilon$.
Empirically, we report $\Delta R^2$ from adding $(S_{t-1},a_{t-1})$.
\end{definition}

\paragraph{Hidden state and SGD dynamics.}
Let the \emph{hidden} state be
\(
x_t \triangleq (\theta_t,\,B_t,\,\tau),
\)
where $\theta_t$ are student parameters, $B_t$ the remaining budget, and $\tau$ task parameters (e.g., margins, noise, imbalances).
Given an action $a_t=(w_t,\ell_t,u_t)$---mixture over datasets, learning rate, and usage fraction---the student samples with \emph{replacement} $m_t=\lfloor u_t B_t\rfloor$ examples from the mixture $\sum_i w_{t,i}\mathcal{D}_i^\tau$ and performs $K_t=\lceil m_t/b\rceil$ SGD steps (batch size $b$) with step size $\ell_t$:
\begin{equation}
\theta_{t+1} \;=\; U(\theta_t; a_t, \tau, \xi_t),\qquad B_{t+1}=B_t-m_t,
\end{equation}
where $\xi_t$ denotes minibatch randomness.
Sampling at $t$ thus depends only on $(x_t,a_t)$.
\textit{Remark:} If drawing \emph{without} replacement from finite pools, augment $x_t$ with per-dataset remaining counts so the process remains Markov; we otherwise assume with-replacement sampling in experiments.
\begin{assumption}[With-replacement sampling]\label{assump:with-repl}
Unless stated otherwise, mini-batches at time $t$ are sampled \emph{with replacement} from the mixture $\sum_i w_{t,i}\mathcal{D}_i^\tau$. This makes the data-generation mechanism conditionally independent of history given $(x_t,a_t)$.
\end{assumption}

\paragraph{Observable state (grouped, task/model/optimizer-agnostic).}
We decompose the observable state into three blocks and concatenate them for control:
\[
S_t \;=\; \Big(\;\underbrace{\{z_{t,i}\}_{i=1}^N}_{\text{dataset-level set}}\;,\;
\underbrace{g^{\text{model}}_t}_{\text{model complexity}}\;,\;
\underbrace{g^{\text{progress}}_t}_{\text{training progress}}\;\Big).
\]
\textbf{(A) Dataset-level DeepSets embeddings.} For each dataset $i$, compute a permutation-invariant embedding
\[
z_{t,i} \;=\; \mathrm{DeepSets}\!\Big(\{\text{probe stats from } \mathcal{D}_i\}\Big)\in\R^{d_z},
\]
where the input set contains only task-agnostic statistics computed on a tiny, fixed probe slice each step:
\begin{itemize}[leftmargin=1.25em]
  \item \emph{Predictive performance \& uncertainty:} macro-Acc/F1 (when labels available), mean NLL (length-normalized for sequences), confidence percentiles $(p_{10},p_{50},p_{90})$, ECE-5.
  \item \emph{Loss/difficulty:} loss histogram or quantiles (8 bins).
  \item \emph{Gradient-based difficulty:} mean/std of per-example gradient norms w.r.t. logits (log-scale).
  \item \emph{Diversity sketch:} embedding variance or SimHash/Sketch counts from a small reservoir.
  \item \emph{Light meta:} $\log$ dataset size and label entropy.
\end{itemize}
All quantities are standardized using robust EMAs (median/IQR) and are architecture- and label-permutation agnostic, following the spirit of \citet{fan2018l2t} (student feedback on samples) while remaining model-agnostic.

\begin{definition}[Grouped state: minimal-viable instantiation]\label{def:minimal}
At each step $t$, the observable grouped state is
$
S_t=\big(\{z_{t,i}\}_{i=1}^N,\,g^{\mathrm{model}}_t,\,g^{\mathrm{progress}}_t\big)
$,
where, for each dataset $i$:
\begin{align*}
z_{t,i} &:= \mathrm{DeepSets}\!\Big(\{\text{loss histogram (8 bins)},\ \text{confidence percentiles }(p_{10},p_{50},p_{90}),\\
&\qquad\ \ \text{ECE-5},\ \text{mean NLL (length-normalized)},\ \text{grad-norm mean/std (log)},\ \text{diversity sketch},\ \log|\mathcal{D}_i|,\ \text{label entropy}\}\Big)\in\R^{d_z},\\
g^{\mathrm{model}}_t &:= \big[\log|\theta|,\ \log\text{FLOPs/forward},\ \text{overfit gap}=\text{loss}_{\text{train}}-\text{loss}_{\text{val}},\ \text{LR},\ \text{momentum or }(\beta_1,\beta_2),\ \text{weight decay}\big],\\
g^{\mathrm{progress}}_t &:= \big[t/H,\ B_t/B_0,\ \mathrm{EMA}(\text{train loss}),\ \mathrm{EMA}(\text{val loss}),\ \Delta\text{loss/step},\ \text{gen gap},\ \text{grad-norm mean/std},\\
&\qquad\ \ \cos\angle(\nabla_t,\nabla_{t-1}),\ \text{LR}/\text{LR}_0\big].
\end{align*}
All scalars are robustly normalized (median/IQR clipping). When labels on the probe are unavailable, macro-Acc/F1 terms are omitted and proper scores (NLL, calibration) are used.
\end{definition}

\textbf{(B) Model-complexity block.} A small vector
\[
g^{\text{model}}_t=\big[\log|\theta|,\;\log\text{FLOPs/forward},\;\text{overfit gap}=\text{loss}_{\text{train}}-\text{loss}_{\text{val}},\;\text{LR},\;\text{momentum or }(\beta_1,\beta_2),\;\text{weight decay}\big].
\]
This captures capacity and optimizer settings without accessing architecture internals.

\textbf{(C) Training-progress block.} A small vector
\[
g^{\text{progress}}_t=\big[t/H,\;B_t/B_0,\;\text{EMA(train loss)},\;\text{EMA(val loss)},\;\Delta\text{loss/step},\;\text{gen gap},\;\text{grad-norm mean/std},\;\cos\angle(\nabla_t,\nabla_{t-1}),\;\text{LR}/\text{LR}_0\big].
\]
These dynamics features approximate the transition-relevant information needed for Markov control.


\paragraph{Confidence calibration (ECE-5).}
We compute a 5-bin expected calibration error on the maximum predicted class probability $p_{\max}$.
Let $\mathcal{B}_b$ be instances with $p_{\max}\in((b-1)/5,b/5]$. Then
$\mathrm{ECE}_5=\sum_{b=1}^5 \frac{|\mathcal{B}_b|}{n}\,\big|\mathrm{acc}(\mathcal{B}_b)-\mathrm{conf}(\mathcal{B}_b)\big|$,
where $\mathrm{conf}(\mathcal{B}_b)$ is the average $p_{\max}$ and $\mathrm{acc}(\mathcal{B}_b)$ the empirical accuracy.
Here $n$ is the number of evaluated instances (tokens for sequences; matched boxes for detection). For regression-style probes, ECE-5 is replaced by a proper calibration score (e.g., empirical coverage error under a Gaussian head).

\begin{proposition}[Hidden-state Markovity]
Under SGD with batches drawn according to $w_t$ \emph{with replacement} and fixed update rule $U$, the process on hidden states satisfies
\(
\Prb(x_{t+1},r_t \mid x_t,a_t)=\Prb(x_{t+1},r_t \mid x_t,a_t,x_{t-1},a_{t-1},\dots).
\)
\end{proposition}
\begin{proof}[Proof sketch]
Sampling at time $t$ depends only on $(x_t,a_t)$; the SGD update is a measurable function of $(\theta_t,a_t,\tau,\xi_t)$.
Therefore $(x_{t+1},r_t)$ depends on history only through $(x_t,a_t)$.
If sampling without replacement, include remaining counts in $x_t$.
\end{proof}

\begin{proposition}[Strict Markovity under information-preserving observation]\label{prop:inject}
If $g$ is injective (i.e., $S_t=g(x_t)$ is an information-preserving encoding), then the observable process is Markov:
\(
\Prb(S_{t+1},r_t \mid S_t,a_t)=\Prb(S_{t+1},r_t \mid S_t,a_t,S_{t-1},a_{t-1},\dots).
\)
\end{proposition}

\paragraph{Reward (macro-accuracy across datasets) and episode termination.}
Let $\mathrm{Acc}_{i,t}$ be validation accuracy on dataset $i$; macro-accuracy $\MacroAcc_t=\frac{1}{N}\sum_{i=1}^N \mathrm{Acc}_{i,t}$.
We use either terminal $r_t=\1\{t=H\}\MacroAcc_H$ or incremental $r_t=\MacroAcc_t-\MacroAcc_{t-1}$.
To avoid reward shaping issues from stochastic early termination, we either forbid early termination (fixed $H$) or use terminal-only rewards and report $\MacroAcc_H$.

\paragraph{Budget as a CMDP.}
Budget consumption $c_t=m_t$ induces a constrained MDP
\(
\max_\pi \ \E[\sum_{t=1}^H r_t]\ \text{s.t.}\ \E[\sum_{t=1}^H c_t]\le B_0.
\)
We optimize the Lagrangian
\(
\mathcal{L}=\E\!\left[\sum_{t=1}^H (r_t-\lambda\,c_t)\right],
\)
with a per-episode dual update
\(
\lambda \leftarrow \big[\lambda + \eta_\lambda\big(\sum_{t=1}^H c_t - B_0\big)\big]_+.
\)

\subsection{Number-of-Datasets Invariance}\label{sec:invariance}
\paragraph{State as a set.}
Let the per-dataset embeddings at time $t$ be $\{z_{t,i}\}_{i=1}^N$.
We require the encoder to be \emph{permutation-invariant} and to admit variable cardinality $N$ without architectural changes or scale drift.
By the DeepSets theorem \citep{zaheer2017deepsets}, any continuous permutation-invariant function $f$ on sets can be written as
\begin{equation}
f(\{z_{t,i}\})=\rho\!\left(\frac{1}{N}\sum_{i=1}^N \phi(z_{t,i})\right),
\label{eq:deepsets}
\end{equation}
for suitable $\phi$ and $\rho$ (universal approximators in practice).
We implement the policy/value encoders as shared per-dataset blocks $\phi$ followed by \emph{mean} pooling and a readout $\rho$, then concatenate $g^{\text{model}}_t$ and $g^{\text{progress}}_t$, making them invariant to dataset order and stable in $N$.

\paragraph{Action as per-dataset logits.}
For mixture control, the policy emits a scalar logit $\alpha_{t,i}$ for each dataset via a \emph{shared} head applied to every $z_{t,i}$:
\(
\alpha_{t,i} = h_{\text{mix}}(\phi(z_{t,i})),\quad
w_{t,i}=\frac{\exp(\alpha_{t,i})\,\1\{n_i>0\}}{\sum_{j=1}^N \exp(\alpha_{t,j})\,\1\{n_j>0\}},
\)
where $n_i$ is the available size for dataset $i$ if sampling without replacement. Under with-replacement sampling,
we drop the mask and set $w_t=\mathrm{softmax}(\{\alpha_{t,i}\})$. Shared heads and normalization over observed elements yield permutation equivariance and well-defined actions for any $N$.

\paragraph{Implication.}
The same learned policy applies to unseen numbers of datasets or new orderings without retraining; only the set $S_t$ changes in size or order, which the DeepSets encoder and per-element policy heads support natively.

\section{Method}
\subsection{Policy and Training}
We use an invariant encoder $f_\psi$ with a shared per-dataset block $\phi_\psi$ and \emph{mean}-pooled DeepSets over $\{z_{t,i}\}$ to produce a representation $h_t$, then concatenate the grouped globals $g^{\text{model}}_t$ and $g^{\text{progress}}_t$. Mean pooling (vs.\ sum) avoids scale drift as $N$ varies.
Policy heads output (i) per-dataset mixture logits $\{\alpha_{t,i}\}$, normalized to $w_t$, (ii) learning rate $\ell_t$ (bounded via squashing to $[\ell_{\min},\ell_{\max}]$), and (iii) usage $u_t\in(0,1)$ (sigmoid).
A value head $V_\psi(S_t)$ shares the invariant encoder and globals.

\paragraph{Action regularizers and barriers.}
To prevent collapse and boundary hugging, we add entropy and barrier regularizers to the PPO loss:
\(
\mathcal{L}_{\text{ent}}= \beta_{\text{mix}}\, H(w_t) + \beta_u\, H(u_t),\quad
\mathcal{L}_{\text{bar}}= \kappa\big[-\log(\ell_t-\ell_{\min})-\log(\ell_{\max}-\ell_t)\big] + \kappa'\big[-\log u_t - \log (1-u_t)\big],
\)
with standard clipping of pre-squash outputs for numerical stability.

\paragraph{Student update semantics (removing confounds).}
We fix batch size $b$.
Usage chooses a sample count $m_t=\lfloor u_t B_t\rfloor$, inducing $K_t=\lceil m_t/b\rceil$ optimizer steps with step size $\ell_t$.
Mini-batches are drawn \emph{with replacement} from the mixture with proportions $w_t$.
This separates the effects of step size and data volume; any residual coupling is tempered by $\mathcal{L}_{\text{bar}}$ and the CMDP penalty.

\begin{algorithm}[H]
\caption{PPO teacher on a Markov, set-valued curriculum MDP (updated)}
\label{alg:ppo}
\begin{algorithmic}[1]
\State Initialize policy/value params $\phi,\psi$
\For{iteration $=1,2,\dots$}
  \For{each parallel episode}
    \State Sample task $\tau\sim\mathcal{T}$; reset student; $B_1\leftarrow B_0$
    \For{$t=1$ to $H$}
      \State \Comment{build grouped state on fixed probes}
      \State For each dataset $i$, compute probe statistics and $z_{t,i}$; set $h_t=\rho\!\big(\tfrac{1}{N}\sum_i \phi(z_{t,i})\big)\,\Vert\,g^{\text{model}}_t\,\Vert\,g^{\text{progress}}_t$
      \State Emit $\{\alpha_{t,i}\}$, $\ell_t$, $u_t$; set $w_t=\mathrm{softmax}(\{\alpha_{t,i}\})$
      \State Apply $K_t$ SGD steps using batches drawn \emph{with replacement} from mixture $w_t$ with step size $\ell_t$
      \State $m_t=\lfloor u_t B_t\rfloor$;\quad $B_{t+1}=B_t-m_t$
      \State (Optional, finite pools) re-mask any dataset with $n_i=0$
      \State Observe $(S_{t+1}, r_t)$
    \EndFor
  \EndFor
  \State Compute advantages; update $\phi,\psi$ with PPO and dual variable $\lambda$ (per-episode update)
\EndFor
\end{algorithmic}
\end{algorithm}


\section{Experiments}
\subsection{Cross-Task Meta-Training and Evaluation}
\paragraph{Task families.}
We instantiate three families for meta-training: (i) classification (synthetic mixtures + small real subsets), (ii) sequence labeling (synthetic span tasks + small NER subsets), and (iii) detection (synthetic shapes with boxes + small real subsets). 
We construct episodes that randomize dataset difficulty, size, and noise. 
The teacher never observes task IDs; it only receives the grouped state (dataset embeddings + model/training blocks).

\paragraph{Leave-one-family-out (LOFO).}
We train the teacher on two families and evaluate zero-shot on the held-out family (e.g., train on classification+NER, test on detection), reporting macro metrics appropriate to each family (accuracy/F1, F1 for NER, mAP/AR for detection), along with \emph{worst-dataset} performance and sample-efficiency AUC.

\paragraph{Large-model transfer.}
We stress-test transfer by deploying the teacher zero-shot to a large ViT-based detector trained over 10 detection datasets. 
The teacher controls dataset mixture, learning rate, and usage; the state is the same grouped observation computed on a fixed probe slice.
We report mAP@[.50:.95], AP$_\mathrm{S/M/L}$, and worst-dataset AP.

\paragraph{Diagnostics.}
We measure (i) \emph{approximate Markovity}: one-step $R^2$ of $S_{t+1}$ from $(S_t,a_t)$ and the $\Delta R^2$ from adding history; (ii) \emph{$N$-invariance}: training at $N\in\{3,5\}$ and evaluating at $N\in\{2,4,7,10\}$ under random permutations; and (iii) entropy/barrier usage to verify stable control.

\paragraph{Baselines.}
Static Uniform, Easy$\to$Hard, Myopic Greedy (one-step gain surrogate), bandits (LinUCB/Thompson over datasets), and schedule controllers (PBT/BOHB).

\subsection{Validating Markovity}\label{sec:markov-exp}
We fit a one-step predictor $\hat{F}_\eta$ trained to minimize $\|S_{t+1}-\hat{F}_\eta(S_t,a_t)\|^2$ on logged rollouts; high $R^2$ supports approximate Markovity of the grouped state (Def.~\ref{def:approx-markov-tight}).
We split train/test \emph{by episode} to avoid leakage and report $R^2$ per feature, alongside a trivial ``no-change'' baseline.
We also test whether augmenting inputs with history $(S_{t-1},a_{t-1})$ improves $R^2$; a small $\Delta R^2$ indicates limited additional information in history.
\begin{table}[H]
\centering
\caption{One-step prediction on the grouped observation (placeholders). Higher is better.}
\label{tab:markov}
\begin{tabular}{lccc}
\toprule
Env & $R^2(S_t,a_t\!\to\!S_{t+1})$ & $\Delta R^2$ w/ history & GRU policy gain \\
\midrule
Gaussian & $0.00$ & $+0.00$ & $+0.00$ \\
Linear & $0.00$ & $+0.00$ & $+0.00$ \\
Shapes & $0.00$ & $+0.00$ & $+0.00$ \\
\bottomrule
\end{tabular}
\end{table}

\subsection{Testing Number-of-Datasets Invariance}
We train teachers with $N\in\{3,5\}$ and evaluate zero-shot on $N\in\{2,4,7,10\}$, as well as under random dataset permutations at each episode start.
We report MacroAcc and permutation robustness $\sigma_\pi(\MacroAcc)$, the standard deviation across permutations of the same episode.
\begin{table}[H]
\centering
\caption{$N$-scaling and permutation robustness across tasks (placeholders).}
\label{tab:ninvariance}
\begin{tabular}{lcccc}
\toprule
Train $N$ & Test $N$ & MacroAcc $\uparrow$ & Worst Acc $\uparrow$ & Permutation $\sigma$ \\
\midrule
3 & 2 & 00.0 & 00.0 & 0.00 \\
3 & 7 & 00.0 & 00.0 & 0.00 \\
5 & 10 & 00.0 & 00.0 & 0.00 \\
\bottomrule
\end{tabular}
\end{table}

\subsection{Main Results (Placeholders)}
\begin{figure}[H]
  \centering
  \fbox{\rule{0pt}{2in}\rule{0.95\linewidth}{0pt}}
  \caption{Learning curves (MacroAcc vs. steps) in cross-task and LOFO evaluations.}
  \label{fig:indist}
\end{figure}

\begin{table}[H]
\centering
\caption{Final MacroAcc (\%) and Worst-dataset Acc (\%) on cross-task and LOFO tests.}
\label{tab:indist}
\begin{tabular}{lcc}
\toprule
Method & MacroAcc $\uparrow$ & Worst-dataset $\uparrow$ \\
\midrule
Static Uniform & 00.0 & 00.0 \\
Easy$\to$Hard & 00.0 & 00.0 \\
Myopic Greedy & 00.0 & 00.0 \\
Bandit (LinUCB) & 00.0 & 00.0 \\
PBT/BOHB & 00.0 & 00.0 \\
\textbf{PPO Teacher (ours)} & \textbf{00.0} & \textbf{00.0} \\
\bottomrule
\end{tabular}
\end{table}

\subsection{Out-of-Distribution Transfer (Placeholders)}
\begin{figure}[H]
  \centering
  \fbox{\rule{0pt}{2in}\rule{0.95\linewidth}{0pt}}
  \caption{Zero-shot large-model transfer (e.g., ViT detection) and OOD regions (higher noise/imbalance).}
  \label{fig:ood}
\end{figure}

\begin{table}[H]
\centering
\caption{Zero-shot and few-shot MacroAcc on OOD grid and held-out task families.}
\label{tab:ood}
\begin{tabular}{lcc}
\toprule
Method & Zero-shot $\uparrow$ & Few-shot (10 eps) $\uparrow$ \\
\midrule
Static Uniform & 00.0 & 00.0 \\
Bandit (Thompson) & 00.0 & 00.0 \\
\textbf{PPO Teacher (ours)} & \textbf{00.0} & \textbf{00.0} \\
\bottomrule
\end{tabular}
\end{table}

\subsection{Ablations (Placeholders)}
\begin{itemize}[leftmargin=1.5em]
\item \textbf{State sufficiency:} remove trends; ablate dataset-embedding features or model/progress blocks (e.g., drop ECE or gradient stats); replace with alternative scalars; switch MLP$\to$GRU.
\item \textbf{Action parameterization:} continuous simplex vs. sparse top-$k$; effect of discretization for $\ell,u$.
\item \textbf{Encoder invariance:} shared per-dataset block + pooling vs. flat concatenation.
\item \textbf{Horizon/budget:} vary $H$ and $B_0$; measure stability and sample efficiency.
\end{itemize}

\section{Discussion}
We unified three design desiderata for RL-based curriculum teaching: \emph{Markovity}, \emph{number-of-datasets invariance}, and \emph{task/architecture invariance}.
By formalizing hidden-state Markovity under SGD and using a \emph{grouped} observation (dataset-level embeddings + model-complexity + training-progress), we obtain an observation that is strictly Markov when information-preserving and empirically \emph{approximately} Markov otherwise (Def.~\ref{def:approx-markov-tight}).
This aligns with the control viewpoint: the teacher steers learning dynamics as reflected in competence and stability signals, not raw parameters or task semantics.

The DeepSets architecture provides a principled route to invariance: representing the dataset block as a set and using shared per-dataset encoders with \emph{mean} pooling guarantees both order- and cardinality-robust processing without scale drift as $N$ varies. Concatenating model and progress blocks gives transitions/rewards predictive sufficiency. Coupling this with shared per-dataset action heads and normalization yields policies that seamlessly handle unseen $N$ and arbitrary permutations.

\paragraph{Implications.}
A compact, task-agnostic grouped observation is sufficient to control learning dynamics across task families and architectures while preserving Markovity and $N$-invariance.
This enables training the teacher on inexpensive toy episodes and transferring to large-scale models and datasets without architectural changes or task-specific interfaces.

\paragraph{Limitations.}
The grouped state is intentionally coarse; it forgoes task-specific diagnostics (e.g., IoU error typing) that can accelerate learning in specialized settings.
Its effectiveness depends on stable computation of NLL and max-probabilities (sequence normalization matters and should be reported in the appendix).
While we show strong transfer, certain domains (e.g., extreme class imbalance or heavy data augmentation regimes) may benefit from adding one or two \emph{task-agnostic} action/constraint heads (e.g., an augmentation-strength tier), which we leave to future work.

\section{Conclusion}
We presented a Markov, number-of-datasets invariant \emph{and task/architecture-invariant} formulation for RL-based curriculum teaching.
Our grouped state (dataset-level embeddings + model-complexity + training-progress) and set-based policy/value encoders make the problem well-posed for RL and scalable across any $N$ and a variety of task families, integrating these properties into the Abstract, Methods, and Discussion.
This establishes a principled foundation for learning to teach balanced, adaptive curricula across heterogeneous data sources and modalities.

\medskip
\noindent\textbf{Reproducibility.}
We will release code, configs, and scripts to reproduce all experiments (including CI-friendly runs), together with fixed probes and seeds.

% ---------- References ----------
\bibliographystyle{plainnat}
\begin{thebibliography}{30}

\bibitem[Bengio et~al.(2009)Bengio, Louradour, Collobert, and Weston]{bengio2009curriculum}
Y.~Bengio, J.~Louradour, R.~Collobert, and J.~Weston.
\newblock Curriculum learning.
\newblock In \emph{Proceedings of the 26th International Conference on Machine Learning (ICML)}, 2009.

\bibitem[Kumar et~al.(2010)Kumar, Packer, and Koller]{kumar2010selfpaced}
M.~P. Kumar, B.~Packer, and D.~Koller.
\newblock Self-paced learning for latent variable models.
\newblock In \emph{Advances in Neural Information Processing Systems (NeurIPS)}, 2010.

\bibitem[Schaul et~al.(2016)Schaul, Quan, Antonoglou, and Silver]{schaul2016prioritized}
T.~Schaul, J.~Quan, I.~Antonoglou, and D.~Silver.
\newblock Prioritized experience replay.
\newblock In \emph{International Conference on Learning Representations (ICLR)}, 2016.

\bibitem[Jaderberg et~al.(2017)Jaderberg, et~al.]{jaderberg2017pbt}
M.~Jaderberg et~al.
\newblock Population based training of neural networks.
\newblock \emph{arXiv:1711.09846}, 2017.

\bibitem[Snoek et~al.(2012)Snoek, Larochelle, and Adams]{snoek2012practical}
J.~Snoek, H.~Larochelle, and R.~P. Adams.
\newblock Practical Bayesian optimization of machine learning algorithms.
\newblock In \emph{Advances in Neural Information Processing Systems (NeurIPS)}, 2012.

\bibitem[Falkner et~al.(2018)Falkner, Klein, and Hutter]{falkner2018bohb}
S.~Falkner, A.~Klein, and F.~Hutter.
\newblock BOHB: Robust and efficient hyperparameter optimization at scale.
\newblock In \emph{International Conference on Machine Learning (ICML)}, 2018.

\bibitem[Finn et~al.(2017)Finn, Abbeel, and Levine]{finn2017maml}
C.~Finn, P.~Abbeel, and S.~Levine.
\newblock Model-agnostic meta-learning for fast adaptation.
\newblock In \emph{International Conference on Machine Learning (ICML)}, 2017.

\bibitem[Duan et~al.(2016)Duan, et~al.]{duan2016rl2}
Y.~Duan et~al.
\newblock RL$^2$: Fast reinforcement learning via slow reinforcement learning.
\newblock \emph{arXiv:1611.02779}, 2016.

\bibitem[Zaheer et~al.(2017)Zaheer, Kottur, Ravanbakhsh, Poczos, Salakhutdinov, and Smola]{zaheer2017deepsets}
M.~Zaheer, S.~Kottur, S.~Ravanbakhsh, B.~Poczos, R.~Salakhutdinov, and A.~Smola.
\newblock Deep Sets.
\newblock In \emph{Advances in Neural Information Processing Systems (NeurIPS)}, 2017.

\bibitem[Lee et~al.(2019)Lee, Lee, Kim, Kosiorek, Choi, and Teh]{lee2019settransformer}
J.~Lee, Y.~Lee, J.~Kim, A.~R. Kosiorek, S.~Choi, and Y.~W. Teh.
\newblock Set Transformer: A Framework for Attention-based Permutation-Invariant Neural Networks.
\newblock In \emph{International Conference on Machine Learning (ICML)}, 2019.

\bibitem[Fan et~al.(2018)]{fan2018l2t}
Fan et~al.
\newblock Learning to Teach.
\newblock \emph{arXiv:1805.03643}, 2018.

\bibitem[Wu et~al.(2018)]{wu2018l2tloss}
Wu et~al.
\newblock Learning to Teach with Dynamic Loss Functions.
\newblock In \emph{Advances in Neural Information Processing Systems (NeurIPS)}, 2018.

\bibitem[Kendall et~al.(2018)]{kendall2018uncertainty}
A.~Kendall, Y.~Gal, and R.~Cipolla.
\newblock Multi-Task Learning Using Uncertainty to Weigh Losses for Scene Geometry and Semantics.
\newblock In \emph{Conference on Computer Vision and Pattern Recognition (CVPR)}, 2018.

\bibitem[Chen et~al.(2018)]{chen2018gradnorm}
Z.~Chen, V.~Badrinarayanan, C.-Y.~Lee, and A.~Rabinovich.
\newblock GradNorm: Gradient Normalization for Adaptive Loss Balancing in Deep Multitask Networks.
\newblock \emph{arXiv:1711.02257}, 2018.

\bibitem[Narvekar et~al.(2020)]{narvekar2020survey}
S.~Narvekar, J.~Sinapov, M.~Leonetti, and P.~Stone.
\newblock Curriculum Learning for Reinforcement Learning: A Survey.
\newblock \emph{arXiv:2003.04960}, 2020.

\end{thebibliography}

\end{document}
